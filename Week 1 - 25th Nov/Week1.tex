\documentclass[handout, aspectratio=169]{beamer}
\mode<presentation>{}
\usepackage[utf8]{inputenc}
\newcommand{\fl}[1]{\left\lfloor #1 \right\rfloor}


\title{MA 109 : Calculus-I\\ D1 T4, Tutorial 1}
\author{Devansh Jain}
\date[25-11-2020]{25th November 2020}
\institute[IITB]{IIT Bombay}
\usetheme{Warsaw}
\usecolortheme{beetle}
\newtheorem{defn}{Definition}
\begin{document}

\begin{frame}
    \titlepage
\end{frame}

\begin{frame}{Questions to be Discussed}
    \begin{itemize}
        \item Sheet 1
            \begin{itemize}
                \item 2 (iv) - Sandwich Theorem for finding limits
                \item 3 (ii) - Checking convergence of a sequence
                \item 5 (iii) - Monotonic Bounded sequences are convergent
                \item 7 - Proof using $\epsilon$ - $n_0$ definition
                \item 9 - Relation between product and convergence
                \item 11 - Conditions for exchanging product and limits (Discussed in lecture)
            \end{itemize}
    \end{itemize}
\end{frame}

\begin{frame}{Tutorial Sheet 1}
	2. (iv) $\displaystyle\lim_{n\to \infty}(n)^{1/n}.$\\~\\
	\uncover<2->{Define $h_n := n^{1/n} - 1.$}\\
	\uncover<3->{Then, $h_n \ge 0 \quad \forall n \in \mathbb{N}.$ } \uncover<4->{(Why?)}\\
	\uncover<5->{Observe the following for $n > 2:$ }\\~\\
	\uncover<6->{$n = (1 + h_n)^n > 1 + nh_n + \dbinom{n}{2}h_n^2$}\uncover<7->{$> \dbinom{n}{2}h_n^2 = \dfrac{n(n-1)}{2}h_n^2.$}\\~\\
	\uncover<8->{Thus, $h_n < \sqrt{\dfrac{2}{n-1}} \quad \forall n > 2.$}\\~\\
	\uncover<9->{Using Sandwich Theorem, we get that $\displaystyle\lim_{n\to \infty}h_n = 0$ which gives us that $\displaystyle\lim_{n\to \infty}n^{1/n} = 1.$ }\\~\\
	\uncover<10->{You may use AM-GM on $(n-1)$ 1s and two $\sqrt{n}$ to get upper bound for $n^{1/n}$. }
\end{frame}

\begin{frame}{Tutorial Sheet 1}
    3. (ii) To show: $\left\{(-1)^n\left(\dfrac{1}{2}-\dfrac{1}{n}\right)\right\}_{n \ge 1}$ is \emph{not} convergent.\\~\\
    \uncover<2->{We will use the following two results: }\uncover<3->{(a) Sum of convergent sequences is convergent. }\uncover<4->{(b) The sequence $\{(-1)^n\}_{n\ge1}$ is not convergent.}\\
    \uncover<5->{We now proceed as follows:}\\
    \uncover<6->{$a_n := (-1)^n\left(\dfrac{1}{2}-\dfrac{1}{n}\right) = \dfrac{(-1)^n}{2} - \dfrac{(-1)^n}{n}.$}\\~\\
    \uncover<7->{It is easy to show that $b_n := \dfrac{(-1)^n}{n}$ is convergent. (Its absolute value will behave the same way as $1/n.$)}\\
    \uncover<8->{Now, for the sake of contradiction, let us assume that $(a_n)$ converges. Then, by (a), we have it that $c_n := a_n + b_n = \dfrac{(-1)^n}{2}$ must be convergent.}\\
    \uncover<9->{However, $(c_n)$ converging is equivalent to $\{(-1)^n\}_{n\ge1}$ converging.} \uncover<10->{(Why?)}\\
    \uncover<10->{However, by (b), we know that the above is false. Thus, we have arrived at a contradiction.}
\end{frame}

\begin{frame}{Tutorial Sheet 1}
	5. (iii) $a_1 = 2,\;a_{n+1} = 3 + \dfrac{a_n}{2} \quad \forall n \ge 1.$\\~\\
	\uncover<2->{Claim 1. $a_n < 6 \quad n \in \mathbb{N}.$ }\\
	\uncover<3->{\emph{Proof.}  We shall prove this via induction. The base case $n = 1$ is immediate as $2 < 6.$}\\
	\uncover<4->{Assume that it holds for $n = k.$ }\\
	\uncover<5->{$a_{k+1} = 3 + \dfrac{a_n}{2} < 3 + \dfrac{6}{2} = 6.$ }\\
	\uncover<6->{By principle of mathematical induction, we have proven the claim. \hfill $\blacksquare$ }\\~\\
	\uncover<7->{Claim 2. $a_n < a_{n+1} \quad \forall n \in \mathbb{N}.$ }\\
	\uncover<8->{\emph{Proof.} $a_{n+1} - a_n = 3 - \dfrac{a_n}{2} = \dfrac{6 - a_n}{2} > 0 \implies a_{n+1} > a_n.$ \hfill $\blacksquare$ }\\~\\
	\uncover<9->{Thus, $(a_n)$ is a monotonically increasing sequence that is bounded above. Therefore, it must converge. Use $\displaystyle\lim_{n\to \infty}a_{n+1} = \lim_{n\to \infty}a_n = L$ and then solve for $L$. }
\end{frame}

\begin{frame}{Tutorial Sheet 1}
    7. If $\displaystyle\lim_{n\to \infty}a_n = L \neq 0,$ show that there exists $n_0 \in \mathbb{N}$ such that \\
    \hspace{10em} $|a_n| \ge \dfrac{|L|}{2} \quad \text{ for all } n \ge n_0.$ \\
    \uncover<2->{Let us choose $\epsilon = \dfrac{|L|}{2}.$} \uncover<3->{(Why is this a valid choice of $\epsilon?$)}\\
    \uncover<4->{By hypothesis, there exists $n_0 \in \mathbb{N}$ such that $|a_n - L| < \epsilon$ whenever $n \ge n_0.$}
    \begin{align*}
        \uncover<5->{&|a_n - L| < \epsilon & \forall n \ge n_0}\\
        \uncover<6->{\implies& ||a_n| - |L|| < \epsilon & \forall n \ge n_0}\\
        \uncover<7->{\implies& -\epsilon < |a_n| - |L| < \epsilon & \forall n \ge n_0}\\
        \uncover<8->{\implies& |L| - \epsilon < |a_n|  & \forall n \ge n_0}\\
        \uncover<9->{\implies& \dfrac{|L|}{2} < |a_n|  & \forall n \ge n_0}
    \end{align*}
\end{frame}

\begin{frame}{Tutorial Sheet 1}
    9. (i) $\{a_nb_n\}_{n\ge1}$ is convergent, if $\{a_n\}_{n\ge1}$ is convergent.\\
    \phantom{9.} (ii) $\{a_nb_n\}_{n\ge1}$ is convergent, if $\{a_n\}_{n\ge1}$ is convergent and $\{b_n\}_{n\ge1}$ is bounded.\\~\\
    \uncover<2->{Both are {\color[rgb]{1, 0, 0} false.}}\\
    \uncover<3->{The sequences, $a_n := 1\quad \forall n \in \mathbb{N}$ and $b_n := (-1)^n \quad \forall n \in \mathbb{N}$ act as a counterexample for both the statements.}
\end{frame}

\begin{frame}{Tutorial Sheet 1}
	11. (i) We shall show that the statement is false with the help of a counterexample.\\
	Let $a = -1, \; b = 1, \; c = 0.$ Define $f$ and $g$ as follows:\\
	$f(x) = x$ and $g(x) = \left\{\begin{array}[h]{c l}
		1/x & ;x \neq 0 \\
		0 & ;x = 0
	\end{array}
	\right. .$\\
	It can be seen that $\displaystyle\lim_{x\to 0}f(x) = 0$ but $\displaystyle\lim_{x\to c}[f(x)g(x)] = \lim_{x\to 0}1 = 1.$\\~\\
	(ii) We shall prove that the given statement is true.\\
	We are given that $g$ is bounded. Thus, $\exists M \in \mathbb{R}^+$ such that $|g(x)| \le M \quad \forall x \in (a,\; b).$\\
	Let $\epsilon > 0$ be given. We want to show that there exists $\delta > 0$ such that $|f(x)g(x) - 0| < \epsilon$ whenever $0 < |x - c| < \delta.$\\
	Let $\epsilon_1 = \epsilon/M.$
	As $\displaystyle\lim_{x\to c}f(x) = 0,$ there exists $\delta > 0$ such that $0 < |x - c| < \delta \implies |f(x)| < \epsilon_1.$\\
	Thus, whenever $0 < |x-c| < \delta,$ we have it that $|f(x)g(x) - 0| = |f(x)||g(x)| \le |f(x)|\cdot M < \epsilon_1\cdot M = \epsilon.$ \hfill $\blacksquare$
\end{frame}

\begin{frame}{Tutorial Sheet 1}
	(iii) We shall prove that the given statement is true.\\
	Let $\epsilon > 0$ be given.\\
	Let $l := \displaystyle\lim_{x\to c}g(x).$\\
	Let $\epsilon_1 = \epsilon/(|l| + \epsilon).$\\~\\
	By hypothesis, there exists $\delta_1 > 0$ such that $0 < |x - c| < \delta_1 \implies |g(x) - l| < \epsilon.$\\
	Also, there exists $\delta_2 > 0$ such that $0 < |x-c| < \delta_2 \implies |f(x)| < \epsilon_1.$\\~\\
	Let $\delta = \min\{\delta_1,\;\delta_2\}.$ Then, whenever $0 < |x - c| < \delta,$ we have that:
	$|f(x)g(x)| = |f(x)g(x) - lf(x) + lf(x)| \le |f(x)||(g(x) - l)| + |l||f(x)| < |f(x)|\epsilon + |l||f(x)| = |f(x)|(\epsilon + |l|) < \epsilon_1(\epsilon + |l|)= \epsilon.$\\
	Thus, we have it that $0 < |x - c| < \delta \implies |f(x)g(x) - 0| < \epsilon.$ \hfill $\blacksquare$\\
\end{frame}

\begin{frame}{References}
    Lecture Slides by Prof. Ravi Raghunathan for MA 109 (Autumn 2020) \\
    Tutorial slides prepared by Aryaman Maithani for MA 105 (Autumn 2019) \\
    Solutions to tutorial problems for MA 105 (Autumn 2019) \\
\end{frame}

\end{document}
