\documentclass[aspectratio=169]{beamer}
\mode<presentation>{}
\usepackage[utf8]{inputenc}
\newcommand{\fl}[1]{\left\lfloor #1 \right\rfloor}


\title{MA 109 : Calculus-I\\ D1 T4, Tutorial 5}
\author{Devansh Jain}
\date[23-12-2020]{23rd December 2020}
\institute[IITB]{IIT Bombay}
\usetheme{Warsaw}
\usecolortheme{beetle}
\newtheorem{defn}{Definition}

\begin{document}

\begin{frame}
    \titlepage
\end{frame}

\begin{frame}{Questions to be Discussed}
    \center\textbf{Start the Recording!} \\~\\
    \begin{itemize}
        \item Sheet 5
            \begin{itemize}
                \item 2 (ii), (iii) - Contours and Level sets
                \item 4 - Continuity of operations on f(x), g(y)
                \item 6 (ii) - Partial derivatives
                \item 8 - Partial derivatives again
                \item 10 - Directional derivatives
            \end{itemize}
    \end{itemize}
\end{frame}

\begin{frame}{Tutorial Sheet 5}
	2. (i) Given any $c$ from the options, the level curve is the line $x - y = c$ in the $XY$ plane, that is, the set of points $\{(x,\;y) \in \mathbb{R}^2 : x - y = c\}$ in $\mathbb{R}^2.$ \\
	The contour line for that $c$ is the line in $\mathbb{R}^3$ which consists of the set of points $\{(x,\;y,\;z)\in\mathbb{R}^3 : x - y = c,\;z = c\}.$ That is, it is the level curve just shifted parallelly in the $z-$direction. \\~\\
	(ii) For $c < 0,$ the contour lines and level curves are empty sets. \\
	For $c = 0,$ the level curve is just the point $(0,\;0)\in \mathbb{R}^2$ and the counter line is $(0,\;0,\;0)\in \mathbb{R}^3.$ \\
	For $c > 0,$ the level curve $L$ is the circle $\{(x,\;y)\in\mathbb{R}^2:x^2 + y^2 = c\}$ and the contour line is the ``same curve, just shifted $c$ units upwards'' in $z-$direction. More precisely, the contour line is the set $L \times \{c\}.$ \\~\\
	(iii) You can work this out similarly. \\
	(You should get a right hyperbola with $x = 0$ and $y = 0$ as asymptotes for $c \ne 0$) \\
\end{frame}
	
\begin{frame}{Tutorial Sheet 5}
	Note: It is technically not correct to say that the contour lines are just the ``level curves shifted upwards'' because the two curves are not lying in the same space. More precisely, $\mathbb{R}^2 \not\subset \mathbb{R}^3.$ However, we do have a natural ``embedding'' of $\mathbb{R}^2$ into $\mathbb{R}^3$ which is what we were referring to.
\end{frame}

\begin{frame}{Tutorial Sheet 5}
	(4) (i), (ii), (iii), (iv) \\
	Let $(x_0,\;y_0)$ be any point in $\mathbb{R}^2.$ We show that the function is continuous at this point. \\
	Let $(x_n,\;y_n)$ be any sequence in $\mathbb{R}^2$ such that $(x_n,\;y_n) \to (x_0,\;y_0).$ This gives us that $x_n \to x_0$ and $y_n \to y_0.$ \hfill (Why?) \\
	Hence, $f(x_n) \to f(x_0)$ and $g(y_n) \to g(y_0).$ (Definition of continuity of real functions.) \\
	Now, we can use properties of sum and difference of real sequences to get our answers. \\~\\
	For (iii), use the fact that $\max\{a,\;b\} = \frac{a + b + |a - b|}{2}$ and that modulus is a continuous function. Similar considerations apply for (iv).
\end{frame}

\begin{frame}{Tutorial Sheet 5}
	I purposefully left the proof for you. There are several ways to write and I don't want to influence on your practice. \\
	You all know how the quiz went and some of you might have realized the lack of practice of writing proofs. \\
	This is a question you can write properly for practice and I will help improve. \\
\end{frame}

\begin{frame}{Tutorial Sheet 5}
	6. (ii) Let $f:\mathbb{R}^2 \to \mathbb{R}$ denote the function given. \\
	Then, 
	\begin{align*}
		f_x(0,\;0) &= \displaystyle\lim_{h\to 0}\frac{f(0+h,\;0) - f(0,\;0)}{h} \\
		&= \displaystyle\lim_{h\to 0}\left(\frac{\sin^2(h)}{h|h|}\right)
	\end{align*}
	The above limit does not exist. \hfill (Why?)
	(Hint: Take a strictly positive sequence and a strictly negative sequence, both of which converge to 0.) \\
	It can be verified that $f_y(0,\;0)$ also does not exist in a similar manner.
\end{frame}

\begin{frame}{Tutorial Sheet 5}
	8. The continuity of $f$ is immediate. It is extremely similar to what we've seen many times by now. \\
	Let us show that the partial derivatives don't exist. \\
	The partial derivative of $f$ at $(0,\;0)$ with respect to the first variable $(x)$ is given by
	\[\lim_{h\to 0}\frac{f(0 + h,\; 0) - f(0,\;0)}{h} = \lim_{h\to 0}\sin\left(\frac{1}{h}\right),\]
	which we know does not exist. \\
	Similar considerations apply for the other partial derivative.
\end{frame}

\begin{frame}{Tutorial Sheet 5}
	10. The continuity of $f$ at $(0, 0)$ is easy to show using the $\epsilon-\delta$ condition.\\
	Indeed, observe that $|f(x, y) - f(0, 0)| = \left|\sqrt{x^2 + y^2}\right|$ for $y \neq 0$ and $|f(x, y) - f(0, 0)| = 0$ for $y = 0.$\\
	Thus, in general, we have that $|f(x, y) - f(0, 0)| \le \left|\sqrt{x^2 + y^2}\right|.$ \\
	Let $\delta := \epsilon$ and call it a day.\\~\\
	%
	For a unit vector $\textbf{u} := (u_1, u_2)$ and $t \neq 0,$
	\[\frac{f\left(0+t u_{1}, 0+t u_{2}\right)-f(0,0)}{t} = \left\{
		\begin{array}{c c}
			0 & u_2 = 0\\
			\frac{u_2}{|u_2|} & u_2 \neq 0
		\end{array}
	\right.\]
	Hence, $\left(\mathbf{D_u} f\right)(0,0)$ exists for all $\textbf{u}.$ Thus, all directional derivatives exist.\\~\\
\end{frame}

\begin{frame}{Tutorial Sheet 5}
	\emph{If} $f$ is differentiable, then the total derivative \emph{must} be $(f_x(0, 0), f_y(0, 0)) = (0, 0).$ Let us now see whether this does indeed satisfy the condition for being the total derivative. For that, we must check whether
	\[\lim _{(h, k) \rightarrow(0,0)} \frac{f\left(0+h, 0+k\right)-f\left(0, 0\right)-0 h-0 k}{\sqrt{h^{2}+k^{2}}}=0.\]
	For $(h,k) \neq (0,0),$ we have it that
	\[\frac{f\left(0+h, 0+k\right)-f\left(0, 0\right)-0 h-0 k}{\sqrt{h^{2}+k^{2}}} = \frac{k}{|k|}.\]
	It is clear that the limit of the above expression as $(h, k) \to (0, 0)$ does not exist. Hence, $f$ is not differentiable at $(0, 0).$
\end{frame}

\begin{frame}{References}
    Lecture Slides by Prof. Ravi Raghunathan for MA 109 (Autumn 2020) \\
    Tutorial slides prepared by Aryaman Maithani for MA 105 (Autumn 2019) \\
    Solutions to tutorial problems for MA 105 (Autumn 2019) \\
\end{frame}

\end{document}
