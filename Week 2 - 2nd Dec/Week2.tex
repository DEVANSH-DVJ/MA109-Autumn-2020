\documentclass[aspectratio=169]{beamer}
\mode<presentation>{}
\usepackage[utf8]{inputenc}
\newcommand{\fl}[1]{\left\lfloor #1 \right\rfloor}


\title{MA 109 : Calculus-I\\ D1 T4, Tutorial 2}
\author{Devansh Jain}
\date[02-12-2020]{2nd December 2020}
\institute[IITB]{IIT Bombay}
\usetheme{Warsaw}
\usecolortheme{beetle}
\newtheorem{defn}{Definition}
\begin{document}

\begin{frame}
    \titlepage
\end{frame}

\begin{frame}{Questions to be Discussed}
    \begin{itemize}
        \item Sheet 1
            \begin{itemize}
                \item 7 - Proof using $\epsilon$ - $n_0$ definition (Optional, Discussed last time)
                \item 10 - Proof of Even-odd convergence (Optional, Talked about last time)
                \item 13 (ii) - Check Continuity of a function
                \item 15 - Check differentiability of a function
                \item 18 - Evaluate derivative of Multiplicative Cauchy functional equation
            \end{itemize}
        \item Sheet 2
            \begin{itemize}
                \item 3 - Intermediate Value Property (IVP) and Rolle's Theorem
                \item 5 - Mean Value Theorem (MVT)
            \end{itemize}
    \end{itemize}
\end{frame}

\begin{frame}{Tutorial Sheet 1}
    7. If $\displaystyle\lim_{n\to \infty}a_n = L \neq 0,$ show that there exists $n_0 \in \mathbb{N}$ such that \\
    \hspace{10em} $|a_n| \ge \dfrac{|L|}{2} \quad \text{ for all } n \ge n_0.$ \\
    Let us choose $\epsilon = \dfrac{|L|}{2}.$ Why is this a valid choice of $\epsilon?$) \\
    By hypothesis, there exists $n_0 \in \mathbb{N}$ such that $|a_n - L| < \epsilon$ whenever $n \ge n_0.$
    \begin{align*}
        &|a_n - L| < \epsilon & \forall n \ge n_0 \\
        \implies& ||a_n| - |L|| < \epsilon & \forall n \ge n_0 \\
        \implies& -\epsilon < |a_n| - |L| < \epsilon & \forall n \ge n_0 \\
        \implies& |L| - \epsilon < |a_n|  & \forall n \ge n_0 \\
        \implies& \dfrac{|L|}{2} < |a_n|  & \forall n \ge n_0
    \end{align*}
\end{frame}

\begin{frame}{Tutorial Sheet 1}
	10. To show:\\
	$\{a_n\}_{n \ge 1}$ is convergent $\iff$ $\{a_{2n}\}_{n \ge 1}$ and $\{a_{2n+1}\}_{n \ge 1}$ converge to the same limit. \\~\\
	\emph{Proof.} $(\implies)$ Let $b_n := a_{2n}$ and $c_n := a_{2n+1}.$ We are given that $\displaystyle\lim_{n\to \infty}a_n = L.$ We must show that $\displaystyle\lim_{n\to \infty}b_n = \lim_{n\to \infty}c_n.$ \\
	Let $\epsilon > 0$ be given. By hypothesis, there exists $n_0 \in \mathbb{N}$ such that $|a_n - L| < \epsilon$ for $n \ge n_0.$ \\
	Note that $2n > n$ and $2n + 1 > n$ for all $n \in \mathbb{N}.$ Thus, we have that \\
	$|b_{n} - L| < \epsilon$ and $|c_n - L| < \epsilon$ for all $n \ge n_0.$ \\
	Thus, $\displaystyle\lim_{n\to \infty}b_n = \lim_{n\to \infty}c_n = L.$ \hfill $\blacksquare$ 
\end{frame}

\begin{frame}{Tutorial Sheet 1}
	$(\impliedby)$ Let $(b_n)$ and $(c_n)$ be as defined before. We are given that $\displaystyle\lim_{n\to \infty}b_n = \displaystyle\lim_{n\to \infty}c_n = L.$ We must show that $(a_n)$ converges.\\
	Let $\epsilon > 0$ be given. By hypothesis, there exists $n_1, \; n_2 \in \mathbb{N}$ such that \\
	$|b_n - L| < \epsilon$ for all $n \ge n_1$ \hfill (1) \\
	 and $|c_n - L| < \epsilon$ for all $n \ge n_2.$ \hfill (2) \\
	Choose $n_0 = \max\{2n_1,\;2n_2+1\}.$ \\
	Let $n \ge n_0$ be even. Then, $n \ge 2n_1$ or $n/2 \ge n_1$ and $a_n = b_{n/2}.$ By (1), we have it that $|a_n - L| < \epsilon.$ \\
	Similarly, let $n \ge n_0$ be odd. Then, $n \ge 2n_2 + 1$ or $(n-1)/2 \ge n_2$ and $a_n = c_{(n-1)/2}.$ By (2), we have it that $|a_n - L| < \epsilon.$ \\
	Thus, we have shown that $|a_n - L| < \epsilon$ whenever $n \ge n_0.$ This is precisely what it means for $(a_n)$ to converge to $L.$ \hfill $\blacksquare$
\end{frame}

\begin{frame}{Tutorial Sheet 1}
	13. (ii) The function is continuous everywhere.\\
	\emph{Proof.} For $x \neq 0,$ it simply follows from the fact that product and composition of continuous functions is continuous.\\
	To show continuity at $x = 0:$\\
	Let $(x_n)$ be any sequence of real numbers such that $x_n \to 0.$ We must show that $f(x_n) \to 0.$\\
	Let $\epsilon > 0$ be given.\\
	Observe that $|f(x_n) - 0| = \left|x_n\sin\left(\dfrac{1}{x_n}\right)\right| \le |x_n|.$\\
	Now, we shall use the fact $x_n \to 0.$ By this hypothesis, there must exist $n_1 \in \mathbb{N}$ such that $|x_n| = |x_n - 0| < \epsilon \quad \forall n \ge n_1.$\\
	Choosing $n_0 = n_1,$ we have it that $|f(x_n) - 0| \le |x_n| < \epsilon \quad \forall n \ge n_0.$ \hfill $\blacksquare$
\end{frame}

\begin{frame}{Tutorial Sheet 1}
	15. For $x \neq 0$, it simply follows from the fact that product and composition of differentiable functions is differentiable. \\
	To show differentiable at $x = 0$ and evaluating $f'(0)$ \\
	$f'(0) = \displaystyle\lim_{h\to 0}\dfrac{f(0+h) - f(0)}{h} = \displaystyle\lim_{h\to 0}\dfrac{h^2 \sin(1/h)}{h} = \displaystyle\lim_{h\to 0}h\sin(1/h)$ \\~\\
	$|f'(0)| \le \displaystyle\lim_{h\to 0} |h| = 0 \implies f'(0) = 0$ \\~\\
	 We can compute $f'(x) = 2x\sin(1/x) - \cos(1/x)$ for $x \neq 0$ and $f'(0) = 0$. \\
	$f'$ is not continuous as limit at $x = 0$ is not defined (Why?). \\
\end{frame}

\begin{frame}{Tutorial Sheet 1}
	18. Given: $f(x + y) = f(x)f(y)$ for all $x, y \in \mathbb{R}.$ \hfill (1)\\
	Let $x = y = 0.$ This gives us that $f(0) = \left(f(0)\right)^2.$\\
	Thus, $f(0) = 0$ or $f(0) = 1.$\\~\\
	Case 1. $f(0) = 0.$\\
	Substitute $y = 0$ in (1). Thus, $f(x) = f(0)f(x) = 0.$\\
	Therefore, $f$ is identically $0$ which means it's differentiable everywhere with derivative $0.$ \\
	Verify that $f'(c) = f'(0)f(c)$ does hold for all $x \in \mathbb{R}.$ (We did not need to use the fact that $f$ is differentiable at $0,$ it followed from definition.)\\~\\
	Case 2. $f(0) = 1.$\\
	As $f$ is differentiable at $0,$ we know that:\\
	$\displaystyle\lim_{h\to 0}\dfrac{f(0+h) - f(0)}{h} = f'(0) \implies \displaystyle\lim_{h\to 0}\dfrac{f(h) - 1}{h} = f'(0).$ \hfill (2)\\
\end{frame}

\begin{frame}{Tutorial Sheet 1}
	Now, let us show that $f$ is differentiable everywhere.\\
	Let $c \in \mathbb{R}.$ We must show that the following limit exists:\\
	$\displaystyle\lim_{h\to 0}\dfrac{f(c + h) - f(c)}{h}$\\~\\
	Using (1), we can write the above expression as:\\
	$\displaystyle\lim_{h\to 0}\dfrac{f(c)f(h) - f(c)}{h} = \lim_{h\to 0}\dfrac{f(c)(f(h) - 1)}{h} = f(c)\cdot\lim_{h\to 0}\dfrac{f(h) - 1}{h}.$\\~\\
	By (2), we know that the above limit exists. Thus, we have it that $f$ is differentiable at $c$ for every $c \in \mathbb{R}.$ Moreover, $f'(c) = f'(0)f(c).$\\~\\
	\textbf{(Optional)} We have gotten that the derivative of $f$ is a scalar multiple of $f.$ Use this to conclude.	
\end{frame}

\begin{frame}{Tutorial Sheet 2}
	3. Part 1. We will first show the existence of such an $x_0 \in (a, b).$\\
	\emph{Proof.} $I := [a, b]$ is an interval and $f$ is continuous. Thus, $f$ has the intermediate value property on $I.$ Thus, the range $J := f(I)$ must be an interval. As $f(a)$ and $f(b)$ are of different signs, $0$ lies between them. As $f(a), f(b) \in J$ and $J$ is an interval, we have it that $0 \in J = f(I).$
	Thus, $0 = f(x_0)$ for some $x_0 \in I = (a, b).$ \hfill $\blacksquare$\\~\\
	Part 2. Now we will show the uniqueness of $x_0.$ Assume that there exists $x_1 \in (a, b)$ such that $f(x_1) = 0.$ We may assume that $x_0 < x_1.$\\
	Now, we know the following:\\
	(i) $f$ is continuous on $[x_0, x_1],$\\
	(ii) $f$ is differentiable on $(x_0, x_1),$ and\\
	(iii) $f(x_0) = f(x_1).$\\
	Thus, by Rolle's Theorem, there exists $x_2 \in (x_0, x_1)$ such that $f'(x_2) = 0.$ But this contradicts the hypothesis that $f'(x) \neq 0$ for all $x \in (a, b).$ \hfill $\blacksquare$\\
\end{frame}

\begin{frame}{Tutorial Sheet 2}
	5. To prove that $|\sin a - \sin b| \le |a - b|$ for all $a, b \in \mathbb{R}.$\\
	Case 1. $a = b.$ Trivial.\\
	Case 2. $a \neq b.$ Without loss of generality, we can assume that $a < b.$\\
	As $f(x) := \sin(x)$ is continuous and differentiable on $\mathbb{R},$ there exists $c \in (a, b)$ such that $f'(c) = \dfrac{f(b) - f(a)}{b - a}.$ \hfill (By MVT)\\~\\
	Also, we know that $|f'(c)| = |\cos c| \le 1.$\\~\\
	Thus, we have it that $\left|\dfrac{f(b) - f(a)}{b - a}\right| \le 1.$\\~\\
	This is equivalent to what we wanted to prove. \hfill $\blacksquare$
\end{frame}

\begin{frame}{References}
    Lecture Slides by Prof. Ravi Raghunathan for MA 109 (Autumn 2020) \\
    Tutorial slides prepared by Aryaman Maithani for MA 105 (Autumn 2019) \\
    Solutions to tutorial problems for MA 105 (Autumn 2019) \\
\end{frame}

\end{document}
