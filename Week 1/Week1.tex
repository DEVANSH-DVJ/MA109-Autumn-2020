\documentclass[handout, aspectratio=169]{beamer}
\mode<presentation>{}
\usepackage[utf8]{inputenc}
\newcommand{\fl}[1]{\left\lfloor #1 \right\rfloor}


\title{MA 109 : Calculus-I\\ D1 T4, Tutorial 01}
\author{Devansh Jain}
\date[25-11-2020]{25th November 2020}
\institute[IITB]{IIT Bombay}
\usetheme{Warsaw}
% \usecolortheme{beetle}
\newtheorem{defn}{Definition}
\begin{document}

\begin{frame}
    \titlepage
\end{frame}

\begin{frame}{Summary}
    \begin{itemize}
        \item Sheet 1
            \begin{itemize}
                \item 1
                \item 3
                \item 6
                \item 7
                \item 8
                \item 9
            \end{itemize}
    \end{itemize}
\end{frame}

\begin{frame}{Expectations}
    What we expect from you, before you come to the tutorial is the following:
    \begin{enumerate}
        \item You have read the lecture slides that have been uploaded up to that tutorial.
        \item You have \textit{attempted} the questions that are to be discussed in the tutorials.
    \end{enumerate}
\end{frame}

\begin{frame}{Sheet 1}
    1. (i) $\displaystyle\lim_{n\to \infty}\dfrac{10}{n} = 0.$\\~\\
    \uncover<2->{Let $\epsilon > 0$ be given.}
    \uncover<3->{We must show that there exists $n_0 \in \mathbb{N}$ such that for all $n \ge n_0,$ the following is true: $\left|\dfrac{10}{n} - 0\right| < \epsilon.$}\\
    \uncover<4->{$\left|\dfrac{10}{n} - 0\right| < \epsilon \iff \dfrac{10}{n} < \epsilon$} \uncover<5->{$\iff \dfrac{10}{\epsilon} < n.$}\\~\\
    \uncover<6->{Let $n_0 = \fl{\dfrac{10}{\epsilon}} + 1.$}
    \uncover<7->{It is clear that $n_0 > \dfrac{10}{\epsilon}.$\\ Moreover, for any $n \ge n_0,$ we will have $n > \dfrac{10}{\epsilon}.$}\\~\\
    \uncover<8->{Thus, we have shown that for every $\epsilon > 0,$ there exists $n_0 \in \mathbb{N}$ such that $\left|\dfrac{10}{n}\right| < \epsilon$ for all $n \ge n_0.$ $\therefore \displaystyle\lim_{n\to \infty}\dfrac{10}{n} = 0.$}
\end{frame}

\begin{frame}{Sheet 1}
    1. (ii) $\displaystyle\lim_{n\to \infty}\dfrac{5}{3n+1} = 0$\\~\\
    \uncover<2->{Let $\epsilon > 0$ be given.}
    \uncover<3->{We must show that there exists $n_0 \in \mathbb{N}$ such that $\left|\dfrac{5}{3n+1} - 0\right| < \epsilon$ for all $n \ge n_0.$}\\
    \uncover<4->{$$\left|\dfrac{5}{3n+1} - 0\right| < \epsilon \iff \dfrac{5}{3n+1}< \epsilon \iff \dfrac{1}{3}\left(\dfrac{5}{\epsilon} - 1\right) < n.$$}
    \uncover<5->{Thus, we can choose any $n_0 > \frac{1}{3}\left(\frac{5}{\epsilon}-1\right).$}\\~\\
    \uncover<6->{One such choice is $n_0 = \max\left\{1, \fl{\frac{1}{3}\left(\frac{5}{\epsilon}-1\right)}\right\}+1.$}\\
    \uncover<7->{Note: The choice of $n_0$ is not unique. Our choice of $n_0$ might not be the smallest but that is okay.}
\end{frame}

\begin{frame}{Sheet 1}
    1. (iii) $\displaystyle\lim_{n\to \infty}\dfrac{n^{2/3}\sin(n!)}{n+1}=0.$\\~\\
    \uncover<2->{Let $\epsilon > 0$ be given.}
    \uncover<3->{We must show that there exists $n_0 \in \mathbb{N}$ such that $\left|\dfrac{n^{2/3}\sin(n!)}{n+1} - 0\right| < \epsilon$ for all $n \ge n_0.$}\\
    \uncover<4->{$\left|\dfrac{n^{2/3}\sin(n!)}{n+1} - 0\right| < \epsilon \iff \left|\dfrac{n^{2/3}\sin(n!)}{n+1}\right| < \epsilon$} \uncover<5->{{\color[rgb]{1, 0, 0} $\impliedby$} $\left|\dfrac{n^{2/3}}{n+1}\right| < \epsilon$}\\~\\
    \uncover<5->{Note the direction of implication of the red arrow. We have used the fact that $|\sin x| < 1$ for all real $x.$}
\end{frame}

\begin{frame}{Sheet 1}
    \uncover<1->{$\left|\dfrac{n^{2/3}}{n+1}\right| < \epsilon \impliedby \left|\dfrac{n^{2/3}}{n}\right| < \epsilon$} \uncover<2->{$\iff \dfrac{1}{n^{1/3}} < \epsilon \iff \dfrac{1}{\epsilon^3} < n.$}\\~\\
    \uncover<3->{Thus, we can choose $n_0 = \fl{\dfrac{1}{\epsilon^3}} + 1.$}\\~\\
    \uncover<4->{By our arrows of implication, it can be seen that for $n \ge n_0,$ the desired inequality holds.}
\end{frame}

\begin{frame}{Sheet 1}
    1. (iv) $\displaystyle\lim_{n\to \infty}\left(\dfrac{n}{n+1} - \dfrac{n+1}{n}\right) = 0$\\
    \uncover<1->{Let $\epsilon > 0$ be given.}
    \uncover<1->{We must show that there exists $n_0 \in \mathbb{N}$ such that $\left|\dfrac{n^{2/3}\sin(n!)}{n+1} - 0\right| < \epsilon$ for all $n \ge n_0.$}\\
    \uncover<2->{Observe the following:}\\
    \uncover<3->{$\left|\dfrac{n}{n+1} - \dfrac{n+1}{n}\right| = \left|1 - \dfrac{1}{n+1} - 1 - \dfrac{1}{n}\right| = \left| - \dfrac{1}{n+1} - \dfrac{1}{n}\right|$}\\
    \uncover<4->{$=\dfrac{1}{n+1} + \dfrac{1}{n} < \dfrac{2}{n}$}\\
    Thus, if we choose $n_0 = \fl{\dfrac{2}{\epsilon}} + 1,$ we have it that the desired inequality holds.
\end{frame}

\begin{frame}{Sheet 1}
    3. (i) To show: $\left\{\dfrac{n^2}{n+1}\right\}_{n \ge 1}$ is \emph{not} convergent.\\~\\
    \uncover<2->{We will use the fact that convergent sequences are bounded. We will try to show that the sequence given is not bounded. That would imply that the sequence does not converge.} \uncover<3->{(Why?)}\\
    \uncover<4->{\[\dfrac{n^2}{n+1} > \dfrac{n^2-1}{n+1} = \dfrac{(n-1)(n+1)}{n+1} = n - 1\]}
    \uncover<5->{Thus, the sequence given is bounded below by $n-1,$ but by Archimedean property, we know that $n-1$ is not bounded above. Thus, our sequence is not bounded (above). As a result, it is not convergent. \hfill $\blacksquare$}
\end{frame}

\begin{frame}{Sheet 1}
    3. (ii) To show: $\left\{(-1)^n\left(\dfrac{1}{2}-\dfrac{1}{n}\right)\right\}_{n \ge 1}$ is \emph{not} convergent.\\~\\
    \uncover<2->{We will use the following two results: }\uncover<3->{(a) Sum of convergent sequences is convergent. }\uncover<4->{(b) The sequence $\{(-1)^n\}_{n\ge1}$ is not convergent.}\\
    \uncover<5->{We now proceed as follows:}\\
    \uncover<6->{$a_n := (-1)^n\left(\dfrac{1}{2}-\dfrac{1}{n}\right) = \dfrac{(-1)^n}{2} - \dfrac{(-1)^n}{n}.$}\\~\\
    \uncover<7->{It is easy to show that $b_n := \dfrac{(-1)^n}{n}$ is convergent. (Its absolute value will behave the same way as $1/n.$)}\\
    \uncover<8->{Now, for the sake of contradiction, let us assume that $(a_n)$ converges. Then, by (a), we have it that $c_n := a_n + b_n = \dfrac{(-1)^n}{2}$ must be convergent.}\\
    \uncover<9->{However, $(c_n)$ converging is equivalent to $\{(-1)^n\}_{n\ge1}$ converging.} \uncover<10->{(Why?)}\\
    \uncover<10->{However, by (b), we know that the above is false. Thus, we have arrived at a contradiction.}
\end{frame}

\begin{frame}{Sheet 1}
    6. Given $\displaystyle\lim_{n\to \infty}a_n = L,$ we need to find $\displaystyle\lim_{n\to \infty}a_{n+1}.$\\
    \uncover<2->{In other words, if we define $b_n := a_{n+1},$ we find the limit of $(b_n),$ if it exists.}\\
    \uncover<3->{Let $\epsilon > 0$ be given. As $(a_n)$ is convergent, there exists $n_1 \in \mathbb{N}$ such that $|a_n - L| < \epsilon$ for all $n \ge n_1.$ \hfill (1)}\\
    \uncover<4->{Choose $n_0 = n_1,$ then, for any $n \ge n_0,$ we have that $|b_n - L| = |a_{n+1} - L| < \epsilon.$ \hfill (2)}\\
    \uncover<5->{The last inequality is due to the following:}\\
    \uncover<6->{$n+1 > n \ge n_0 = n_1$ and using (1).}\\
    \uncover<7->{Thus, by (2), we have shown that $\displaystyle\lim_{n\to \infty}b_n = \lim_{n\to \infty}a_{n+1} = L.$}
\end{frame}

\begin{frame}{Sheet 1}
    6. Given $\displaystyle\lim_{n\to \infty}a_n = L,$ we need to find $\displaystyle\lim_{n\to \infty}|a_n|.$\\
    \uncover<2->{Like before, let us define $b_n := |a_n|.$} \uncover<3->{It seems reasonable to guess that the limit must $|L|,$ let us try to prove that.}\\
    \uncover<4->{Let $\epsilon > 0$ be given. As $(a_n)$ is convergent, there exists $n_1 \in \mathbb{N}$ such that $|a_n - L| < \epsilon$ for all $n \ge n_1.$ \hfill (1)}\\
    \uncover<5->{Choose $n_0 = n_1,$ then, for any $n \ge n_0,$ we have that $|b_n - |L|| = |\;|a_n| - |L|\;| \le |a_n - L| < \epsilon.$ \hfill (2)}\\
    \uncover<6->{The last inequality is due to the following:}\\
    \uncover<6->{$|\;|x| - |y|\;| \le |x - y|$ for all $x,\;y \in \mathbb{R}$ and using (1).}\\
    \uncover<7->{Thus, by (2), we have shown that $\displaystyle\lim_{n\to \infty}b_n = \lim_{n\to \infty}|a_n| = L.$}
\end{frame}

\begin{frame}{Sheet 1}
    7. If $\displaystyle\lim_{n\to \infty}a_n = L \neq 0,$ show that there exists $n_0 \in \mathbb{N}$ such that
    \[|a_n| \ge \dfrac{|L|}{2} \quad \text{ for all } n \ge n_0.\]
    \uncover<2->{Let us choose $\epsilon = \dfrac{|L|}{2}.$} \uncover<3->{(Why is this a valid choice of $\epsilon?$)}\\
    \uncover<4->{By hypothesis, there exists $n_0 \in \mathbb{N}$ such that $|a_n - L| < \epsilon$ whenever $n \ge n_0.$}\\
    \begin{align*}
        \uncover<5->{&|a_n - L| < \epsilon & \forall n \ge n_0}\\
        \uncover<6->{\implies& ||a_n| - |L|| < \epsilon & \forall n \ge n_0}\\
        \uncover<7->{\implies& -\epsilon < |a_n| - |L| < \epsilon & \forall n \ge n_0}\\
        \uncover<8->{\implies& |L| - \epsilon < |a_n|  & \forall n \ge n_0}\\
        \uncover<9->{\implies& \dfrac{|L|}{2} < |a_n|  & \forall n \ge n_0}
    \end{align*}
\end{frame}

\begin{frame}{Sheet 1}
    8. If $a_n \ge 0$ and $\displaystyle\lim_{n\to \infty}a_n = 0,$ show that $\lim_{n\to \infty}a_n^{1/2} = 0.$\\
    \uncover<2->{Let $\epsilon>0$ be given. This means that $\epsilon^2 > 0.$ }\\
    \uncover<3->{By hypothesis, there exists $n_0 \in \mathbb{N}$ such that $|a_n - 0| = a_n < \epsilon^2$ for all $n \ge n_0.$ }\\
    \uncover<4->{Thus, $|a_n^{1/2} - 0| = a_n^{1/2} < \epsilon$ for all $n \ge n_0.$ }\\
    \uncover<5->{By definition of limit, we have shown that $\displaystyle\lim_{n\to \infty}a_n^{1/2} = 0.$ \hfill $\blacksquare$}\\~\\
    \uncover<6->{At what place(s) did we use that $a_n \ge 0?$}\\~\\
    \uncover<7->{Hint for \textbf{optional:} Use the inequality $\left|\sqrt[n]{a} - \sqrt[n]{b}\right| \le \sqrt[n]{|a - b|}$ for $n \in \mathbb{N}.$ }
\end{frame}

\begin{frame}{Sheet 1}
    9. (i) $\{a_nb_n\}_{n\ge1}$ is convergent, if $\{a_n\}_{n\ge1}$ is convergent.\\
    \phantom{9. } (ii) $\{a_nb_n\}_{n\ge1}$ is convergent, if $\{a_n\}_{n\ge1}$ is convergent and $\{b_n\}_{n\ge1}$ is bounded.\\~\\
    \uncover<2->{Both are {\color[rgb]{1, 0, 0} false.}}\\
    \uncover<3->{The sequences, $a_n := 1\quad \forall n \in \mathbb{N}$ and $b_n := (-1)^n \quad \forall n \in \mathbb{N}$ act as a counterexample for both the statements.}
\end{frame}

\begin{frame}{References}
    Lecture Slides by Prof. Ravi Raghunathan for MA 109 (Autumn 2020) \\
    Tutorial slides prepared by Aryaman Maithani for MA 105 (Autumn 2019) \\
    Solutions to tutorial problems for MA 105 (Autumn 2019) \\
\end{frame}

\end{document}
