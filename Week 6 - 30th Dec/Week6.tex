\documentclass[aspectratio=169]{beamer}
\mode<presentation>{}
\usepackage[utf8]{inputenc}
\newcommand{\fl}[1]{\left\lfloor #1 \right\rfloor}


\title{MA 109 : Calculus-I\\ D1 T4, Tutorial 6}
\author{Devansh Jain}
\date[30-12-2020]{30th December 2020}
\institute[IITB]{IIT Bombay}
\usetheme{Warsaw}
\usecolortheme{beetle}
\newtheorem{defn}{Definition}

\newcommand{\R}{\mathbb{R}}


\begin{document}

\begin{frame}
    \titlepage
\end{frame}

\begin{frame}{Questions to be Discussed}
    \center\textbf{Start the Recording!} \\~\\
    \begin{itemize}
        \item Sheet 6
            \begin{itemize}
                \item 2 - Directional Derivative in $\R^2$
                \item 4 - Directional Derivative in $\R^3$
                \item 5 - Mixed Partial Derivative
                \item 8 - Maxima, Minima and Saddle Points
                \item 9 - Absolute Maximum and Minimum
            \end{itemize}
    \end{itemize}
\end{frame}

\begin{frame}{Tutorial Sheet 6}
    2. $f(x,y)=x^2+sin(xy)$ \\
    First, as $f_x = 2x + y \cos(xy)$ and $f_y = x \cos(xy)$ are continuous, we note that $f(x,y)$ is differentiable at the point $(1,0)$. \\
    This implies that, for a unit vector $\mathbf{u}$, we have $D_{\mathbf{\underline{u}}} f(1,0)= ((\nabla f)(1,0)) \cdot \mathbf{u}$. \\ 
    We have to find a u = $(u_1, u_2)$ so that $D_{\mathbf{\underline{u}}} f(1,0) = 1 $. \\
    Plugging in x=1 and y=0 in $f_x$ and $f_y$ , we get that $(\nabla f)(1,0)=(2,1)$ \\
    Thus, we want $(2,1) \cdot (u_1, u_2) = 1$, that is $2u_1 + u_2 = 1 \implies 2u_1 = 1 - u_2$. \\
    As $\mathbf{u}$ is a unit vector, we also have $u_1^2+u_2^2=1$ \\
    $2u_1=1-u_2 \implies 4u_1^2=u_2^2-2u_2+1 \implies 4-4u_2^2=u_2^2-2u_2+1 \implies 5u_2^2-2u_2-3=0$ \\
    $\implies u_2=1$ or $u_2=\frac{-3}{5}.$ \\
    The corresponding unit vectors are u=$(0,1)$ and u=$(4/5, -3/5)$ \\
    Indeed we see that u $=(0,1)$ and u $=(4/5, -3/5)$ do imply $D_{\mathbf{\underline{u}}} f(1,0) = 1 $. \\
    Thus, these are the only directions where directional derivative takes the value 1. \\
    Basically we have proved that $D_{\mathbf{\underline{u}}} f(1,0) = 1 \iff $ u $=(0,1)$ and u $=(4/5, -3/5)$.
\end{frame}

\begin{frame}{Tutorial Sheet 6}
	4. It is not too tough to show that the direction of the normal to a sphere at a point on the sphere is the same as the direction of the vector joining the center to that point. \\
	Indeed, we get that $(\nabla S)(x_0, y_0, z_0) = 2(x_0, y_0, z_0),$ where $S(x, y, z) := x^2 + y^2 + z^2$ for $(x, y, z) \in \mathbb{R}^3.$ \\
	Thus, the required $\mathbf{u}$ is $\frac{1}{3}(2, 2, 1).$ \\
	Hence,
	\begin{equation*}
	    (\mathbf{D_\underline{u}}F)(2, 2, 1) = \lim_{t\to 0}\frac{3(2t/3) - 5(2t/3) + 2(t/3)}{t} = -\frac{2}{3}.
	\end{equation*}
\end{frame}

\begin{frame}{Tutorial Sheet 6}
	5. We shall assume that $z$ is a ``sufficiently smooth'' function of $x$ and $y.$ \\
	We are given that $\sin (x+y)+\sin (y+z)=1$ and $\cos (y+z) \neq 0.$ \\ 
	Differentiating with respect to $x$ while keeping $y$ constant gives us $\cos (x+y)+\cos (y+z) \frac{\partial z}{\partial x}=0.$ \hfill $(*)$ \\~\\
	Similarly, differentiating with respect to $y$ while keeping $x$ constant gives us $\cos (x+y)+\cos (y+z)\left(1+\frac{\partial z}{\partial y}\right)=0.$ \hfill $(**)$ \\~\\
	Differentiating $(*)$ with respect to $y$ gives us $-\sin (x+y) - \sin (y+z) \left( 1 + \frac{\partial z}{\partial y} \right) \frac{\partial z}{\partial x} + \cos (y+z) \frac{\partial^{2} z}{\partial x \partial y} = 0.$
	\footnote{Note that I have implicitly assumed that $\frac{\partial^2z}{\partial x\partial y} = \frac{\partial^2z}{\partial y\partial x}.$ However, using a different set of calculations, one can arrive at the same answer without assuming this. I encourage you to try that.}
\end{frame}

\begin{frame}{Tutorial Sheet 6}
	Thus, using $(*)$ and $(**),$ we get
	\begin{equation*}
	    \begin{aligned}
	        \frac{\partial^{2} z}{\partial x \partial y}
	            &=\frac{1}{\cos (y+z)}\left[\sin (x+y)+\sin (y+z) \cdot\left(1+\frac{\partial z}{\partial y}\right) \frac{\partial z}{\partial x}\right] \\~\\
	            &=\frac{1}{\cos (y+z)}\left[\sin (x+y)+\sin (y+z)\left(-\frac{\cos (x+y)}{\cos (y+z)}\right)\left(-\frac{\cos (x+y)}{\cos (y+z)}\right)\right] \\~\\
	            &=\frac{\sin (x+y)}{\cos (y+z)}+\tan (y+z) \frac{\cos ^{2}(x+y)}{\cos ^{2}(y+z)} 
	    \end{aligned}
	\end{equation*}
\end{frame}

\begin{frame}{Tutorial Sheet 6}
	8. (i) $f(x, y)=\left(x^{2}-y^{2}\right) e^{-\left(x^{2}+y^{2}\right) / 2}.$ \\
	Note that the above function is defined on $D = \mathbb{R}^2.$ \\
	Thus, every point is an interior point of $D.$ Moreover, it can be seen that the partial derivatives of all orders exist and are continuous everywhere. \\
	For $(x_0, y_0)$ to be a point of extrema or a saddle point, it must be the case that $(\nabla f)(x_0, y_0) = (0, 0).$ \\~\\
	Note that $f_x(x, y) =x e^{1 / 2\left(-x^{2}-y^{2}\right)}\left(-x^{2}+y^{2}+2\right).$ \\
	Also, $f_y(x, y) =y e^{1 / 2\left(-x^{2}-y^{2}\right)}\left(-x^{2}+y^{2}-2\right).$ \\~\\
	Thus, solving $(\nabla f)(x_0, y_0) = (0, 0)$ gives us that $(x_0, y_0) \in \{(0, 0),\;(0, \sqrt{2}),\;(0, -\sqrt{2}),\;(-\sqrt{2},0),\;(\sqrt{2}, 0)\}.$ \\
	Now, we determine the exact nature using the determinant test.
\end{frame}

\begin{frame}{Tutorial Sheet 6}
	Recall that $(\Delta f)\left(x_{0}, y_{0}\right):=f_{x x}\left(x_{0}, y_{0}\right) f_{y y}\left(x_{0}, y_{0}\right)-f_{x y}\left(x_{0}, y_{0}\right)^{2}.$ \\
	Hence, in our case,
	\begin{equation*}
	    (\Delta f)(x, y) = -e^{-x^{2}-y^{2}} \left(x^{6}-x^{4} y^{2}-3 x^{4}-x^{2} y^{4}+22 x^{2} y^{2}-8 x^{2}+y^{6}-3 y^{4}-8 y^{2}+4\right)
	\end{equation*}
	Moreover, $f_{xx}(x, y) = e^{-\left(x^{2}+y^{2}\right) / 2}(x^4 - x^2y^2 - 5x^2 + y^2 + 2)$ \\
	For $(x_0, y_0) = (0, 0),$ it is clear that it is a saddle point for $f$ as discriminant is $-4 < 0.$ \\~\\
	Note that if $x = 0,$ the discriminant reduces to $-e^{-y^2}(y^6 - 3y^4 -8y^2 + 4).$ \\
	Substituting $y = \pm\sqrt{2}$ gives us that the discriminant is positive with $f_{xx}$ positive and hence, the points are points of local minima. \\~\\
	Similarly, we get that the points $(\pm\sqrt{2}, 0)$ are points of local maxima as they have discriminant positive and $f_{xx}$ negative. 
\end{frame}

\begin{frame}{Tutorial Sheet 6}
	8. (ii) $f(x, y)=f(x, y)=x^{3}-3 x y^{2}.$ \\
	Note that the above function is defined on $D = \mathbb{R}^2.$ \\
	Thus, every point is an interior point of $D.$ Moreover, it can be seen that the partial derivatives of all orders exist and are continuous everywhere. \\
	For $(x_0, y_0)$ to be a point of extrema or a saddle point, it must be the case that $(\nabla f)(x_0, y_0) = (0, 0).$ \\~\\
	Note that $f_x(x, y) = 3x^2 - 3y^2.$ \\
	Also, $f_y(x, y) = -6xy.$ \\~\\
	Thus, solving $(\nabla f)(x_0, y_0) = (0, 0)$ gives us that $(x_0, y_0) = (0, 0).$ \\
	Now, we determine the exact nature using the determinant test.
\end{frame}

\begin{frame}{Tutorial Sheet 6}
	Recall that $(\Delta f)\left(x_{0}, y_{0}\right):=f_{x x}\left(x_{0}, y_{0}\right) f_{y y}\left(x_{0}, y_{0}\right)-f_{x y}\left(x_{0}, y_{0}\right)^{2}.$\\
	Hence, in our case,
	\[(\Delta f)(x_0, y_0) = -36(x_0^2 + y_0^2).\] 
	Thus, for $(x_0, y_0) = (0, 0),$ we get the discriminant is $0.$ \\
	Hence, we get that the discriminant test is { inconclusive!} \\
	This means that we must turn to some other analytic methods of determining the nature. \\~\\
	Now, we note that $f(\delta, 0) = \delta^3$ for all $\delta \in \mathbb{R}.$ \\
	Thus, given any $\epsilon > 0,$ choose $\delta = \pm \epsilon/2.$ \\
	This gives us that $(0, 0)$ is saddle point.
\end{frame}

\begin{frame}{Tutorial Sheet 6}
	9. To find: Absolute maxima and minima of $f(x, y)=\left(x^{2}-4 x\right) \cos y \text { for } 1 \leq x \leq 3,-\pi / 4 \leq y \leq \pi / 4.$\\
	Note that the domain is a closed and bounded set. As $f$ is continuous on the domain, $f$ does achieve a maximum and a minimum.
	 Note that $f_x(x, y) = (2 x-4) \cos y$ and $f_y(x, y) = -\left(x^{2}-4 x\right) \sin y$ for interior points $(x, y).$ \\
	Thus, the only critical point is $p_1 = (2, 0).$ \\~\\
	Now we restrict ourselves to the boundaries to find the local extrema. \\
	``Right boundary:'' This is the line segment $x = 3, -\pi / 4 \leq y \leq \pi / 4.$ \\
	The function now reduces to $-3\cos y$ on this segment. \\
	Using our theory from one-variable calculus, we get that we need to check the points $(3, 0),\;(3, \pi/4),\;(3, -\pi/4).$ \\~\\
	Similar consideration of the ``left boundary'' gives us the points $(1, 0),\;(1, \pi/4),\;(1, -\pi/4).$
\end{frame}

\begin{frame}{Tutorial Sheet 6}
	Now, we look at the ``top boundary.''\\
	The function there reduces to $\frac{x^2 - 4x}{\sqrt{2}}.$ \\
	Once again, using our theory from one-variable calculus, we get that we need to check the points $(1, \pi/4),\;(2, \pi/4),\;(3, \pi/4).$ \\~\\
	Similarly, checking the ``bottom boundary'' gives us the points $(1, -\pi/4),\;(2, -\pi/4),\;(3, -\pi/4).$ \\
	We now tabulate our results as follows:
	\begin{equation*}
	    \begin{array}{|c||c|c|c|c|c|}
        	\hline
        	(x_0, y_0) & (2, 0) & (3, 0) & (3, \pi/4) & (2, \pi/4) & (1, \pi/4) \\
        	\hline
        	f(x_0, y_0) & -4 & -3 & \dfrac{-3}{\sqrt{2}} & \dfrac{-4}{\sqrt{2}} & \dfrac{-3}{\sqrt{2}} \\
        	\hline
        	\hline
        	(x_0, y_0) & (1, 0) & (1, -\pi/4) & (2, -\pi/4) & (3, -\pi/4) &  \\
        	\hline
        	f(x_0, y_0) & -3 & \dfrac{-3}{\sqrt{2}} & \dfrac{-4}{\sqrt{2}} & \dfrac{-3}{\sqrt{2}} & \\
        	\hline 
    	\end{array}
	\end{equation*}
	Thus, we get that $f_{\text{min}} = -4$ at $(2, 0)$ and $f_{max} = -\frac{3}{\sqrt{2}}$ at $(1, \pm \pi/4)$ and $(3, \pm\pi/4).$
\end{frame}

\begin{frame}{References}
    Lecture Slides by Prof. Ravi Raghunathan for MA 109 (Autumn 2020) \\
    Tutorial slides prepared by Aryaman Maithani for MA 105 (Autumn 2019) \\
    Solutions to tutorial problems for MA 105 (Autumn 2019) \\
\end{frame}

\end{document}
