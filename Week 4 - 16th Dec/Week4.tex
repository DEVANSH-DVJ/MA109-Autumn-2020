\documentclass[aspectratio=169]{beamer}
\mode<presentation>{}
\usepackage[utf8]{inputenc}
\newcommand{\fl}[1]{\left\lfloor #1 \right\rfloor}


\title{MA 109 : Calculus-I\\ D1 T4, Tutorial 4}
\author{Devansh Jain}
\date[16-12-2020]{16th December 2020}
\institute[IITB]{IIT Bombay}
\usetheme{Warsaw}
\usecolortheme{beetle}
\newtheorem{defn}{Definition}

\begin{document}

\begin{frame}
    \titlepage
\end{frame}

\begin{frame}{Questions to be Discussed}
    \begin{itemize}
        \item Sheet 4
            \begin{itemize}
                \item 2 (a), (b) -
                \item 3 (ii), (iv) - 
                \item 4 (b) (i), (ii) - 
                \item 6 - 
            \end{itemize}
    \end{itemize}
\end{frame}

\begin{frame}{Tutorial Sheet 4}
	2. (a) Let $P = \{x_0, x_1, \ldots, x_n\}$ be any partition of $[a, b].$\\
	For $i = 1, 2, \ldots, n,$ we have
	\[M_i(f) = \sup_{x \in [x_{i-1}, x_i]}f(x) \ge 0.\]
	We have used that the supremum of a set of non-negative real numbers is nonnegative. (Why?)\\
	Thus, $U(P, f) \ge 0.$ As $f$ is given to be Riemann integrable on $[a, b],$ there exists a sequence $(P_n)$ of partitions of $[a, b]$ such that $U(P_n, f) \to \displaystyle\int_{a}^{b} f(x) dx .$ But $U(P_n, f) \ge 0$ for all $n.$ (Shown above)\\
	Thus, $\displaystyle\int_{a}^{b} f(x) dx = \lim_{n\to \infty}U(P_n, f) \ge 0.$\\
	Note that here we have used the fact that the limit of a sequence of nonnegative real numbers, if it exists, is nonnegative.
\end{frame}

\begin{frame}{Tutorial Sheet 4}
	To prove the next part, let us prove the contrapositive. That is, if $f(x) \neq 0$ for some $x \in [a, b],$ then $\displaystyle\int_{a}^{b} f(x) dx \neq 0.$\\~\\
	%
	Suppose $c \in [a, b]$ is the number such that $f(c) \neq 0.$ As $f(x) \ge 0$ for all $x \in [a, b],$ we have it that $f(c) > 0.$ Let $\epsilon := f(c).$\\
	%Let us assume that $c$ is an interior point 
	As $f$ is continuous, there is a $\delta > 0$ such that if $x \in [a, b]$ and $|x - c| < \delta,$ then $|f(x) - f(c)| < \epsilon/2$ which implies that $\epsilon/2 < f(x).$\\
	Now, let us take the partition $P := \{x_0, x_1, x_2, x_3\}$ with $x_0 = a,\; x_1 = c-\delta,\;x_2 = c+\delta$ and $x_3 = b.$ If it is the case that $x_1 < x_0,$ then discard $x_1.$ If it is the case that $x_2 > x_3,$ discard $x_2.$ Relabel if required.\\
	Now, there exists $x_i \in P$ such that $\displaystyle\inf_{x \in [{x_{i-1}, x_i}]}f(x) \ge \epsilon/2 > 0.$\\
	Thus, $L(P, f) > 0.$ As $f$ is Riemann integrable, $\displaystyle\int_{a}^{b} f(x) dx = \sup\{L(P, f): P \text{ is a partition of } [a, b]\} > 0$ as we have found a partition that has a strictly positive lower sum.
\end{frame}

\begin{frame}{Tutorial Sheet 4}
	2. (b) Let $a = 0, b = 2$ and $f:[a, b] \to \mathbb{R}$ be defined as
	\[f(x) = \left\{\begin{array}[h]{c l}
		0 & ; x \neq 1 \\
		1 & ; x = 1
	\end{array}
	\right.\]
	Show that $f$ is actually Riemann integrable on $[0, 2]$ with the integral equal to $0.$
\end{frame}

\begin{frame}{Tutorial Sheet 4}
	3. (ii) Note that \\
	\[S_n = \sum_{i=1}^{n}\dfrac{n}{i^2 + n^2} = \sum_{i=1}^{n}\dfrac{1}{\left(\frac{i}{n}\right)^2 + 1}\left(\frac{i}{n} - \frac{i-1}{n}\right) .\]
	Define $f:[0, 1] \to \mathbb{R}$ by $f(x) := \tan^{-1}x.$ Then, we have that $f'(x) = \frac{1}{x^2 + 1}.$\\
	As $f'$ is continuous and bounded, it is (Riemann) integrable. \\
	For $n \in \mathbb{N},$ let $P_n := \{0, 1/n, \ldots, n/n\}$ and $t_i := i/n$ for $i = 1, 2, \ldots, n.$\\
	Then, $S_n = R(P_n, f').$ Since $||P_n|| = 1/n \to 0,$ it follows that
	\[R(P_n, f') \to \int_{0}^{1} \dfrac{1}{x^2 + 1} dx = \int_{0}^{1} f'(x) dx. \]
	By the Fundamental Theorem of Calculus (Part 2), we have it that
	\[\lim_{n\to \infty}S_n = \int_{0}^{1} f'(x) dx = f(1) - f(0) = \dfrac{\pi}{4}.\]
\end{frame}

\begin{frame}{Tutorial Sheet 4}
	3. (iv) Note that \\
	\[S_n = \dfrac{1}{n}\sum_{i=1}^{n}\cos\left(\dfrac{i\pi}{n}\right) = \sum_{i=1}^{n}\cos\left(\dfrac{i\pi}{n}\right)\left(\frac{i}{n} - \frac{i-1}{n}\right) .\]
	Define $f:[0, 1] \to \mathbb{R}$ by $f(x) := \frac{1}{\pi}\sin(\pi x).$ Then, we have that $f'(x) = \cos(\pi x).$\\
	As $f'$ is continuous and bounded, it is (Riemann) integrable. \\
	For $n \in \mathbb{N},$ let $P_n := \{0, 1/n, \ldots, n/n\}$ and $t_i := i/n$ for $i = 1, 2, \ldots, n.$\\
	Then, $S_n = R(P_n, f').$ Since $||P_n|| = 1/n \to 0,$ it follows that
	\[R(P_n, f') \to \int_{0}^{1} \cos(\pi x) dx = \int_{0}^{1} f'(x) dx. \]
	By the Fundamental Theorem of Calculus (Part 2), we have it that
	\[\lim_{n\to \infty}S_n = \int_{0}^{1} f'(x) dx = f(1) - f(0) = 0.\]
\end{frame}

\begin{frame}{Tutorial Sheet 4}
	4. (b) Let $u$ and $v$ be differentiable functions defined on appropriate domains.\\
	Let $g$ be a continuous function. Define $G(x) := \displaystyle\int_{a}^{x} g(t) dt.$ Then $G'(x) = g(x),$ by Fundamental Theorem of Calculus (Part 1). Note that
	\[\int_{u(x)}^{v(x)} g(t) dt = \int_{a}^{v(x)} g(t) dt - \int_{a}^{u(x)} g(t) dt = G(v(x)) - G(u(x)).\]
	Thus, by the Chain Rule, one has
	\[\dfrac{d}{dx}\int_{u(x)}^{v(x)} g(t) dt = G'(v(x))v'(x) - G'(u(x))u'(x) = g(v(x))v'(x) - g(u(x))u'(x).\]
	We can now easily solve the question.
\end{frame}
	
\begin{frame}{Tutorial Sheet 4}
	(i)\\
	Given, $F(x) = \displaystyle\int_{1}^{2x} \cos(t^2) dt $
	\begin{align*}
		\therefore \frac{dF}{dx} &= \cos\left((2x)^2\right)(2x)' - \cos(1)(1)' \\
		&= 2\cos(4x^2).
	\end{align*}
\end{frame}

\begin{frame}{Tutorial Sheet 4}
	(ii)\\
	Given, $F(x) = \displaystyle\int_{0}^{x^2} \cos(t) dt $
	\begin{align*}
		\therefore \frac{dF}{dx} &= \cos\left(x^2\right)(x^2)' - \cos(0)(0)'\\
		& = 2x\cos(x^2).
	\end{align*}
\end{frame}

\begin{frame}{Tutorial Sheet 4}
	6. 
	\begin{align*}
		g(x) &= \dfrac{1}{\lambda}\int_{0}^{x} f(t)\sin \lambda(x - t) dt\\
		&= \dfrac{1}{\lambda}\int_{0}^{x} f(t) \left(\sin \lambda x\cos \lambda t - \cos \lambda x \sin \lambda t\right) dt\\
		&= \frac{1}{\lambda}\sin\lambda x\int_{0}^{x} f(t)\cos \lambda t dt - \frac{1}{\lambda}\cos \lambda x \int_{0}^{x} f(t)\sin \lambda t dt \\
	\end{align*}
	Now, we can differentiate $g$ using product rule and Fundamental Theorem of Calculus (Part 1).
	\begin{align*}
	\therefore g'(x) %&= \cos\lambda x\int_{0}^{x} f(t)\cos \lambda t dt + \frac{1}{\lambda}f(x)\sin\lambda x \cos \lambda x dt + \sin \lambda x \int_{0}^{x} f(t)\sin \lambda t dt - \frac{1}{\lambda}f(x)\cos \lambda x \sin \lambda \\
	&= \cos\lambda x\int_{0}^{x} f(t)\cos \lambda t dt + \sin \lambda x \int_{0}^{x} f(t)\sin \lambda t dt \\
	\end{align*}
\end{frame}
		
\begin{frame}{Tutorial Sheet 4}
	It is easy to verify that both $g(0)$ and $g'(0)$ are 0.\\
	We can differentiate $g'$ in a similar way and get,
	\begin{align*}
		g''(x) &= -\lambda\sin\lambda x\int_{0}^{x} f(x)\cos \lambda t dt + f(x)\cos^2\lambda x + \lambda \cos \lambda x \int_{0}^{x} f(t)\sin \lambda t dt \\
		& + f(x)\sin^2 \lambda x\\
		&= f(x) - \lambda^2\left(\dfrac{1}{\lambda}\int_{0}^{x} f(t) \left(\sin \lambda x\cos \lambda t - \cos \lambda x \sin \lambda t\right) dt\right)\\
		&= f(x) - \lambda^2g(x)\\
		\implies & g''(x) + \lambda^2g(x) = f(x) & \blacksquare
	\end{align*}
\end{frame}

\begin{frame}{References}
    Lecture Slides by Prof. Ravi Raghunathan for MA 109 (Autumn 2020) \\
    Tutorial slides prepared by Aryaman Maithani for MA 105 (Autumn 2019) \\
    Solutions to tutorial problems for MA 105 (Autumn 2019) \\
\end{frame}

\end{document}
