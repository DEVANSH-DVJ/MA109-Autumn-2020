\documentclass[aspectratio=169]{beamer}
\mode<presentation>{}
\usepackage[utf8]{inputenc}
\newcommand{\fl}[1]{\left\lfloor #1 \right\rfloor}


\title{MA 109 : Calculus-I\\ D1 T4, Tutorial 3}
\author{Devansh Jain}
\date[09-12-2020]{9th December 2020}
\institute[IITB]{IIT Bombay}
\usetheme{Warsaw}
\usecolortheme{beetle}
\newtheorem{defn}{Definition}
\begin{document}

\begin{frame}
    \titlepage
\end{frame}

\begin{frame}{Questions to be Discussed}
    \begin{itemize}
        \item Sheet 2
            \begin{itemize}
                \item 8 (ii), (iii) Finding Functions with given Conditions
                \item 10 (i) Sketching curve with given properties
                \item 11 Sketching curve with given properties
            \end{itemize}
        \item Sheet 3
            \begin{itemize}
                \item 1 (ii) Taylor Series for $\arctan x$
                \item 2 Taylor Series for a polynomial
                \item 4 Convergence of Maclaurin Series of $e^x$
                \item 5 Integration using Taylor Series
            \end{itemize}
    \end{itemize}

\end{frame}

\begin{frame}{Tutorial Sheet 2}
    8. (ii). $f''(x) > 0$ for all $x \in \mathbb{R}$, $f'(0) = 1, f'(1) = 2$ \\
    \medskip
    $f''(x) > 0$ for all $x \in \mathbb{R}$ \\
    \medskip
    We know such a curve!!! \\
    \medskip
    Can try quadratic $\implies$ Let $f(x)=ax^2+bx+c$ \\
    $f'(x)=2ax+b$ \hspace{30pt} $f''(x)=2a$ \\
    \medskip
    $f''(x) > 0$  $\implies a>0$ \\
    $f'(0) = 1$ $\implies b=1$ \\
    $f'(1)=2$ $\implies 2a+b=2$ \\
    \medskip
    One such function is: $f(x)=\frac{x^2}{2}+x$ \\
\end{frame}

\begin{frame}{Tutorial Sheet 2}
    8. (iii). $f''(x) \geq 0$ for all $x \in \mathbb{R}$, $f'(0) = 1, f(x) <100$ for all $x>0$ \\
    \medskip 
    \textbf{Idea.}
    Derivative at 0 is positive. \\
    $f''$ non negative, means derivative not decreasing, always positive after 0. \\
    $f$ must be strictly increasing.\\
    But f is bounded above. Contradiction!!! \\
    Such a function cannot exist! \\
    Let's write a formal argument. \\
\end{frame}

\begin{frame}{Tutorial Sheet 2}
    8. (iii). \textbf{Proof}. \\
    $f''(x) \geq 0$  \hspace{5pt} $ \forall x \in \mathbb{R} \implies $ $f'$ not decreasing, so $\forall x>0, f'(x)\geq 1$ \\
    To prove: $f$ must exceed 100. \\
    \medskip
    Can't use integration here :( \\
    MVT to the rescue!! \\
    \medskip
    Can write $\dfrac{f(x)-f(0)}{x-0} = f'(c) \geq 1$ $\because c \geq 0$ \\
    Thus, $f(x)\geq x+f(0)$ \\
    \smallskip
    Just take $x=101-f(0) \implies f(x) \geq 101$ . \\
    Done!!!
\end{frame}

\begin{frame}{Tutorial Sheet 3}
    1. (ii). Taylor Series for arctan x. \\
    Can do by integration, as given in paragraph after question 4.
\end{frame}

\begin{frame}{Sheet 3}
    2. Taylor Series of $f(x)=x^3-3x^2+3x-1 = (x-1)^3$ \\
    Can already guess!!!
    \smallskip
    $f(x)=x^3-3x^2+3x-1$ $\implies f(1)=0$  \\
    \smallskip
    $f'(x)=3x^2-6x+3$ $\implies f'(1)=0$\\
    \smallskip
    $f''(x)=6x-6$   $\implies f''(1)=0$\\
    \smallskip
    $f'''(x)=6$ $\implies f'''(1)=6$\\
    \smallskip
    $f^{(n)}(x)=0$ for all $n>3$. 
    \medskip
    Thus, the Taylor series is \\
    $P(x)=\displaystyle \sum_{n=0}^{\infty}\dfrac{f^{(n)}(x_0)}{n!}(x-x_0)^n$ 
    Here, only the third derivative is non-zero! Only one term in the Taylor Series! 
    $P(x)=\dfrac{f'''(0)}{3!}(x-1)^3=\dfrac{6}{6}(x-1)^3=(x-1)^3$
\end{frame}

\begin{frame}{Sheet 3}
    4. Proving convergence of the series $\displaystyle \sum_{k=0}^{\infty} \frac{x^k}{k!}$. \\
    \medskip
    We'll follow steps given in the question.
    Choose $N>2x$. 
    \medskip
    We see that for all $n>N$,  \hspace{10pt}
    $\dfrac{x^{n+1}}{(n+1)!} < \dfrac{1}{2}\cdot\dfrac{x^n}{n!}$
    Let us denote the partial sums of of the given series by $s_m(x)$. 
    We should show that for every $\epsilon>0$, there is a $N \in \mathbb{N}$, such that for all $m, n > N $, $|s_m(x)-s_n(x)|<\epsilon$.
    \medskip
    For this, observe that (assuming WLOG $m>n$), 
    $ \displaystyle  |s_m(x)-s_n(x)| = \left|\sum_{k=n+1}^{m} \dfrac{x^k}{k!}\right| \leq \left|\dfrac{x^n}{n!}\right|\left(\dfrac{1}{2} +\dfrac{1}{4} + ... + \dfrac{1}{2^{m-n}} \right) \leq \left|\dfrac{x^n}{n!}\right|\left(\dfrac{1}{2} +\dfrac{1}{4} + ... \right) \leq \dfrac{|x^n|}{n!} $
\end{frame}

\begin{frame}{Sheet 3}
    4 contd. \\
    We have, \hspace{5pt} $ \displaystyle  |s_m(x)-s_n(x)| \leq 2 \cdot \dfrac{|x^n|}{n!} $ for all $m > n$ \\
    We also have, \\
    for $n>N>2x,$  \hspace{5pt} $\dfrac{x^{n+1}}{(n+1)!} < \dfrac{1}{2}\cdot\dfrac{x^n}{n!} < \dfrac{x^n}{n!} < \dfrac{x^N}{N!} $
\end{frame}

\begin{frame}{Sheet 3}
    5. To integrate:
    $\displaystyle \int \dfrac{e^x}{x} dx$ \\
    $e^x=\displaystyle \sum_{n=0}^{\infty} \dfrac{x^n}{n!}$  $\implies$  \\
    \smallskip
    $\displaystyle \int \dfrac{e^x}{x} dx =  \displaystyle \int \sum_{n=0}^{\infty} \dfrac{x^{n-1}}{n!}dx$
    $=\displaystyle  \sum_{n=0}^{\infty} \dfrac{1}{n!} \int x^{n-1} dx$
    $=\displaystyle  \sum_{n=0}^{\infty} \dfrac{1}{n!} (n-1)x^n$
    $=\displaystyle  \sum_{n=0}^{\infty} \dfrac{(n-1)x^n}{n!} $
\end{frame}

\begin{frame}{References}
    Lecture Slides by Prof. Ravi Raghunathan for MA 109 (Autumn 2020) \\
    Tutorial slides prepared by Krushnakant Bhattad for MA 109 (Autumn 2020) \\
    Solutions to tutorial problems for MA 105 (Autumn 2019) \\
\end{frame}

\end{document}
